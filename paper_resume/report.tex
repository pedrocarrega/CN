%\documentclass[11pt,a4paper,runningheads]{llncs}
\documentclass[11pt,a4paper]{article}
%encoding
%--------------------------------------
\usepackage[T1]{fontenc}
\usepackage[utf8]{inputenc}
%--------------------------------------

%Portuguese-specific commands
%--------------------------------------
\usepackage[portuguese]{babel}
%--------------------------------------

%Hyphenation rules
%--------------------------------------
\usepackage{hyphenat}
\hyphenation{mate-mática recu-perar}
%--------------------------------------

\usepackage{graphicx}
\usepackage{comment}
\usepackage{pgfplots}
\usepackage{amsmath,amssymb,amsfonts}
\usepackage{algorithmic}
\usepackage{graphicx}
\usepackage{textcomp}
\usepackage{xcolor}
\usepackage{adjustbox}
\usepackage{float}


\begin{document}

\title{Fase 3 - Requisitos, Casos de Uso e Arquitetura}
\author{Pedro Carrega, nº49480 \and
Vasco Ferreira, nº49470 \and Ye Yang, nº 49521
}

%\institute{Departamento de Informática da Faculdade de Ciências da Universidade de Lisboa
%\email{\{fc49480,fc49470,fc49521\}@alunos.fc.ul.pt}}

\maketitle

\section{Lançamento em Kubernetes}
O scripts de deployment do sistema foram separados em 4 devido à necessidade dos clusters e node groups estarem ativos, antes de proceder aos próximos passos. Existe também uma secção de edição de ficheiro manual, o que fez a quebra entre o terceiro e o quarto script.

Antes da execução dos scripts é necessário verificar a existência dos seguintes repositórios, roles e policies e caso existam, precisam de ser \textbf{eliminados}:
\begin{enumerate}
	\item Repositórios com nomes \textit{products} e \textit{events}
	\item AWS Role com nome eksServiceRole
	\item AWS Policy com nome ALBIngressControllerIAMPolicyEcommerce
\end{enumerate}

Para a execução dos scripts são necessárias as seguintes ferramentas:
\begin{itemize}
	\item AWS CLI
	\item eksctl
	\item kubectl
\end{itemize}

A região a escolher para o lançamento poderá ser qualquer um, porém recomendamos a região eu-west-1. Esta região terá de ser a mesma nos argumentos de todos os scripts que requeiram a mesma.


\subsection{deploy1.sh}
O primeiro script de deployment recebe 3 argumentos na seguinte ordem:
\begin{enumerate}
	\item A região onde a Stack e o Cluster vão ser lançados (ex.: \textbf{eu-west-1})
	\item O nome da Stack que terá de ser único (nenhuma outra Stack na CloudFormation da conta pessoal poderá ter o mesmo nome) para o script funcionar corretamente (ex.:\textbf{ecommerce-stack})
	\item O nome do Cluster que também terá de ser único (ex.:\textbf{ecommerce-cluster})
\end{enumerate}

A criação da stack e do cluster irá demorar cerca de 10 a 20 minutos até ficarem ativos, após o qual poderemos proceder à execução do segundo script. O estado do script pode ser verificado com o seguinte comando:
\begin{itemize}
	\item \textit{aws eks describe-cluster --name CLUSTER\_NAME} , mudando CLUSTER\_NAME para o nome do cluster dado nos argumentos
\end{itemize}
A execução do segundo script só deve ser feita quando o estado do cluster estiver em \textbf{ACTIVE}.

\subsection{deploy2.sh}
O segundo script recebe os mesmos argumentos que o primeiro, todos na mesma ordem. Neste script vão ser criados os node groups e o pull das imagens dos serviços.

Para realizar o pull, irá ser pedido para inserir os credenciais IAM que dão acesso aos repositórios por nós criados. Estes credenciais encontram-se no ficheiro \textbf{credenciais.txt} juntamente com a região onde os repositórios se encontram.

No fim

\end{document}