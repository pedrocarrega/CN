\documentclass[runningheads]{llncs}

%encoding
%--------------------------------------
\usepackage[T1]{fontenc}
\usepackage[utf8]{inputenc}
%--------------------------------------

%Portuguese-specific commands
%--------------------------------------
\usepackage[portuguese]{babel}
%--------------------------------------

%Hyphenation rules
%--------------------------------------
\usepackage{hyphenat}
\hyphenation{mate-mática recu-perar}
%--------------------------------------

\usepackage{graphicx}
\usepackage{comment}
\usepackage{pgfplots}
\usepackage{amsmath,amssymb,amsfonts}
\usepackage{algorithmic}
\usepackage{graphicx}
\usepackage{textcomp}
\usepackage{xcolor}
\usepackage{adjustbox}


\begin{document}

% \thanks{Supported by organization x.}
\title{Fase 3 - Requisitos, Casos de Uso e Arquitetura}
\author{Pedro Carrega \and
Vasco Ferreira \and Ye Yang
}
%\authorrunning{F. Author et al.}

\institute{Departamento de Informática da Faculdade de Ciências da Universidade de Lisboa
\email{\{fc49480,fc49470,fc49521\}@alunos.fc.ul.pt}}

\maketitle

\section{Requisitos}

\begin{table}[htbp]
	\caption{Details of the hardware used in the experiments}
	\begin{center}
		\begin{tabular}{|p{4cm}|p{8cm}|}
		\hline
			\textbf{\textit{Requisitos Funcionais}} & \textbf{\textit{Descrição}}\\ \hline
			\textbf{\textit{Categorias Disponiveis}} & \textbf{\textit{Serviço que fornece aos clientes todas as categorias}} \\ \hline
			\textbf{\textit{Popularidade das\newline Marcas}} & \textbf{\textit{Serviço que fornece cada marca associada com a sua popularidade}} \\ \hline
			\textbf{\textit{Número de Vendas\newline Individuais}} & \textbf{\textit{Fornece o numero total de vendas de cada marca}} \\ \hline
			\textbf{\textit{Preço Médio de Venda}} & \textbf{\textit{Fornece o preço médio dos produtos vendidos de uma determinada marca}} \\ \hline
			\textbf{\textit{Rácio de Tipo\newline de Eventos}} & \textbf{\textit{Fornece a percentagem de cada tipo de evento}} \\ \hline
	\end{tabular}
	\label{tab1}
	\end{center}
\end{table}

\begin{table}[htbp]
	\caption{Details of the hardware used in the experiments}
	\begin{center}
		\begin{tabular}{|p{3.5cm}|p{9.5cm}|}
		\hline
			\textbf{\textit{Requisitos\newline Não Funcionais}} & \textbf{\textit{Descrição}}\\ \hline
			\textbf{\textit{Portabilidade}} & \textbf{\textit{Implementação de servidor em NodeJS e cliente em JAVA de modo a facilitar o processo de mudança de plataforma do serviço}} \\ \hline
			\textbf{\textit{Legibilidade}} & \textbf{\textit{A separação clara entre as camadas de apresentação, lógica de negócio e acesso à base de dados irá tornar o fluxo do sistema mais legível para os desenvolvedores}} \\ \hline
			\textbf{\textit{Estabilidade}} & \textbf{\textit{A distribuição do sistema por diversas máquinas virtuais, que poderão estar distribuídas por diferentes fornecedores cloud e em diferentes Data Centers permitem obter um sistema estável}} \\ \hline
			\textbf{\textit{Elasticidade}} & \textbf{\textit{O sistema deverá ser capaz de se adaptar à carga de trabalho através do provisionamento e desprovisionamento dos recursos de forma autónoma. Idealmente, de forma que em qualquer ponto do tempo, o sistema apenas utilize o número de máquinas necessárias de forma a corresponder à carga atual}} \\ \hline
			\textbf{\textit{Escalabilidade}} & \textbf{\textit{Capacidade do sistema lidar com o crescimento de carga de trabalho. Pode-se associar à capacidade de elasticidade do sistema}} \\ \hline
			\textbf{\textit{Confiabilidade}} & \textbf{\textit{Visto o sistema estar implementado na núvem, conseguimos garantir alta confiabilidade nos sistema, pois se um servidor no datacenter falhar, conseguimos facilmente migrar a VM para outro servidor funcional}} \\ \hline
	\end{tabular}
	\label{tab1}
	\end{center}
\end{table}
\section{Casos de Uso}




\end{document}